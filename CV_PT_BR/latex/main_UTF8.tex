%%%%%%%%%%%%%%%%%%%%%%%%%%%%%%%%%%%%%%%%%
% Twenty Seconds Resume/CV
% LaTeX Template
% Version 1.0 (14/7/16)
%
% Original author:
% Carmine Spagnuolo (cspagnuolo@unisa.it) with major modifications by 
% Vel (vel@LaTeXTemplates.com) and Harsh (harsh.gadgil@gmail.com)
%
% License:
% The MIT License (see included LICENSE file)
%
%%%%%%%%%%%%%%%%%%%%%%%%%%%%%%%%%%%%%%%%%

%----------------------------------------------------------------------------------------
%	PACKAGES AND OTHER DOCUMENT CONFIGURATIONS
%----------------------------------------------------------------------------------------

\documentclass[a4paper]{twentysecondcv} % a4paper for A4
\usepackage[utf8]{inputenc}
\usepackage[portuguese]{babel}

% Command for printing skill overview bubbles
% Programming skill bars
\programming{{Microsoft Office $-$ LibreOffice $-$ G Suite / 5},  {Linux $-$ \LaTeX $-$  Raspberry Pi/ 4}, {Assembly $-$ SQL $-$  Python/ 2}, {Git $-$ Linux Server $-$ Confluence /4},{ADALM-PLUTO $-$ Freedom Board $-$ ARM / 3}, {Java $-$ C \slash C++ $-$ MATLAB \slash Octave / 4}, {VHDL $-$ TCL $-$ Shell Scripting / 4}, {Arduino $-$ 8051 $-$ 8086 / 3}, {Mentor Graphics $-$ Synopsys $-$ Eagle / 2}, {Xilinx Vivado $-$ Altera Quartus $-$ FPGA / 4}, {Electric VLSI $-$ Doxygen $-$ TaskJuggler / 4},{Cadence $-$ LTspice $-$ SpiceOpus / 4}}

\Languages{{Italiano / 1}, {Francês / 1.3}, {Inglês / 6}, {Português (Nativo) / 6}}

%\uCuP{{Arduino $-$ 8051 $-$ 8086 / 3}, {Mentor Graphics $-$ Synopsys $-$ Fritzing / 3}, {Xilinx Vivado $-$ Altera Quartus $-$ FPGA / 4}, {Electric VLSI $-$ Doxygen $-$ TaskJuggler / 5},{Cadence $-$ LTspice $-$ SpiceOpus / 5}}


% Projects text
\education{
\textbf{Graduate Certificate, Embedded Systems}\\
University of Colorado - Boulder\\
2020 - Em Andamento | Colorado, USA

\textbf{Beng, Electrical Engineering and Biomedical Engineering}\\
Programa de Intercâmbio - Ciência Sem Fronteiras \\
University of Glasgow \\
2014 - 2015 | Glasgow, Scotland

\textbf{BEng., Engenharia Elétrica} \\
Universidade de Brasília \\
2011 - 2019 | Brasília, Brazil
}

%----------------------------------------------------------------------------------------
%	 PERSONAL INFORMATION
%----------------------------------------------------------------------------------------
% If you don't need one or more of the below, just remove the content leaving the command, e.g. \cvnumberphone{}

\cvname{Guilherme \\Shimabuko} % Your name
\cvjobtitle{ Engenheiro Eletricista } % Job
% title/career

\cvnumberphone{+55 61 9 9841-4243} % Phone number
\cvmail{gshimabuko@ieee.org} % Email address

%----------------------------------------------------------------------------------------

\begin{document}

\makeprofile % Print the sidebar
 
%----------------------------------------------------------------------------------------
%	 EXPERIENCE
%----------------------------------------------------------------------------------------
\section{Sobre: }
\small{Engenheiro Eletricista com forte interesse em eletrônica e computação. Entusiasta das filosofias de desenvolvimento de software livre e de código e hardware abertos, com experiência de longa data em diversas distribuições Linux. Pró-ativo, trabalha bem em equipe, com experiência em gestão e treinamento de equipes.}

\section{Experiência Relevante}
\begin{twenty} % Environment for a list with descriptions
\twentyitem
    	{LPCI - Departamento de Engenharia Elétrica - UnB}
		{Assistente de Pesquisa \hfill \textit{Janeiro de 2020 - Atualmente}\\}
        {Estagiário \hfill\textit{Novembro de 2018 - Dezembro de 2019}}
        {
            \color{pblue}{\hspace{8pt}-\hspace{4pt}} \color{black}\textbf{Sistemas Digitais: }Design e validação de sistemas digitais para FPGA e     ASIC para comunicação satelital utilizando ferramentas Cadence.
            
            \color{pblue}{\hspace{8pt}-\hspace{4pt}} \color{black}\textbf{Tecnologia da Informação: }Validação e manutenção das ferramentas de projeto e gestão de servidores linux, bem como suporte para os clientes (Ubuntu, CentOS, ArchLinux).
            
            \color{pblue}{\hspace{8pt}-\hspace{4pt}} \color{black}\textbf{Recursos Humanos: }Seleção e treinamento de equipe para formação de equipe de projeto de circuitos integrados.
            
            \color{pblue}{\hspace{8pt}-\hspace{4pt}} \color{black}\textbf{Gestão de Projetos: }Elaboração de planejamento de projeto. Acompanhamento de atividades. Implantação de sistema de log de atividades. Documentação de projeto. Implantação, otimização e automação de fluxo de projeto.
        }
        \\
	\twentyitem
    	{DFchip}
		{Estagiário}
        {Novembro de 2015 - Janeiro de 2017}
        {
            \color{pblue}{\hspace{8pt}-\hspace{4pt}} \color{black}{\textbf{Sistemas Digitais: }Design e validação de sistemas sigitais para FPGA,     ASIC e SoC, bem como implantação e documentação do fluxo de projetos de sistemas       digitais utilizando ferramentas Cadence.}
            
            \color{pblue}{\hspace{8pt}-\hspace{4pt}} \color{black}{\textbf{Sistemas Analógicos: }Treinamento em design de circuitos analógicos         utilizando ferramentas Cadence.}
            
            \color{pblue}{\hspace{8pt}-\hspace{4pt}} \color{black}{\textbf{Tecnologia da Informação: }Validação e manutenção das ferramentas de         projeto e gestão de sistema Linux (OpenSUSE).}
        }
        \\
    \twentyitem
    	{Medialab – Instituto de Artes – UnB}
		{Estagiário}
		{Fevereiro de 2014 - Agosto de 2014 e Julho de 2012 – Março de 2013}
        {
            \color{pblue}{\hspace{8pt}-\hspace{4pt}} \color{black}\textbf{Prototipagem: }Desenvolvimento de protótipo de dispositivos eletrônicos para exibições de Arte e Tecnologia.
        }
        \\
    \twentyitem
    {LPCI - Departamento de Engenharia Elétrica - UnB}
    {Estagiário}
    {Abril de 2013 - Janeiro de 2014}
    {
        \color{pblue}{\hspace{8pt}-\hspace{4pt}} \color{black}\textbf{Prototipagem: }Desenvolvimento de protótipo de dispositivos eletrônicos para exibições de Arte e Tecnologia.
    }
    \\
    \twentyitem
    {ENETEC Consultoria Junior - Empresa Junior de Engenharia Elétrica}
    {Diretor de Projetos}
    {Julho de 2012 – Julho de 2013}
    {\color{pblue}{\hspace{8pt}-\hspace{4pt}} \color{black}\textbf{Gestão de Projetos: }Elaboração e implantação de metodologias de gerenciamento de projetos. Negociação e gerenciamento de projetos.}
    \\
    \twentyitem
    {ENETEC Consultoria Junior - Empresa Junior de Engenharia Elétrica}
    {Consultor \hfill Janeiro de 2012 - Julho de 2012\\}
    {Trainee  \hfill Julho de 2011 – Dezembro de 2011}
    {\color{pblue}{\hspace{8pt}-\hspace{4pt}} \color{black}\textbf{Consultoria: }Elaboração de projetos de Instalações Elétricas. Criação de Processos para a Diretoria de Recursos Humanos.}
\end{twenty}

%----------------------------------------------------------------------------------------
%	 Lecturing Experience
%----------------------------------------------------------------------------------------
\section{Atividades de Docência}
\begin{twenty}

    \twentyitemshortii{Monitor de Introdução ao Projeto de Circuitos Integrados}{Departamento de Engenharia Elétrica - UnB}{0/2017 e 1/2018 - 2/2018}
    \twentyitemshortii{Monitor de Laboratório de Eletrônica 2}{Departamento de Engenharia Elétrica - UnB}{1/2017 - 2/2017}
    \twentyitemshortii{Instrutor de Curso de Arduíno}{Centro Acadêmico de Engenharia Elétrica - UnB}{Março de 2017 - Abril de 2017}
    \twentyitemshortii{Monitor de Laboratório de Eletrônica 1}{Departamento de Engenharia Elétrica - UnB}{1/2016}
    \twentyitemshortii{Monitor de Sistemas Digitais 2}{Departamento de Engenharia Elétrica - UnB}{1/2014}
    \twentyitemshortii{Monitor de Laboratório de Sistemas Digitais 1}{Departamento de Engenharia Elétrica - UnB}{1/2013}
    \twentyitemshortii{Instrutor de Curso de Excel}{ENETEC Consultoria Junior - Empresa Junior de Engenharia Elétrica - UnB}{Agosto de 2012}
\end{twenty}

\newpage
\secsidebar
\section{Projetos e Publicações}
\begin{twenty}
\twentyitem
{Projeto de Pesquisa Para Parceiro Industrial:}
{'Desenvolvimento de Sistema de Comunicação Satelital baseado no padrão DVB-S2X'}
{}
{\hspace{8pt} Esse projeto visou o desenvolvimento de um sistema de comunicação satelital móvel em parceria com uma empresa privada. A equipe da qual fiz parte foi responsável pela revisão e revalidação da implementação em VHDL do protótipo desenvolvido pela equipe de FPGA, bem como a adaptação desses códigos para o framework de circuitos integrados da Cadence e para os padrões de codificação do \textit{LPCI}. 

\hspace{8pt}Atuei como \textit{Design Leader}, treinando o grupo e gerenciando a equipe, revisando as entregas e projetando partes do sistema. O foco da etapa do projeto em que participei foi o desenvolvimento dos códigos de correção de erro, especialmente o algoritmo BCH. Como parte das minhas atividades, coorientei extra-oficialmente o trabalho de conclusão de curso do aluno Thiago Queiroz Holanda sob a tutela do Professor José Camargo da Costa. Esse trabalho foi apresentado em Dezembro de 2019 sob o título \textit{"Comparison Between the Implementations of a BCH DVB-S2X Decoder in FPGA and in ASIC"}. Durante esse período, também fui responsável pela gestão de TI do projeto. Por fim, tive uma breve participação na equipe de prototipagem em FPGA, trabalhando com a compilação cruzada de códigos C para um processador ARM.

\hspace{8pt}\textit{Habilidades utilizadas:} \textbf{VHDL, ASIC, FPGA, Cadence Tools, Vivado, Shell Scripting, TCL, MATLAB, Octave, Linux, Sistemas Embarcados, Compilação Cruzada, Buildroot, ARM, ADALM-PLUTO, SDR, Sistemas Digitais, Processamento Digital de Sinais.\\}}


\twentyitem
{Colaboração em Projeto de Conclusão de Curso em Machine Learning e finanças:}
{'Aplicação de inteligência computacional no mercado financeiro'}
{}
{\hspace{8pt} Como assistente de pesquisa do LPCI, colaborei com o desenvolvimento do trabalho de conclusão de curso do aluno Tainan Soares Rodrigues, orientado pelo professor Alexandre Romariz e apresentado em Dezembro de 2019. Esse projeto visou a avaliação da eficácia de um sistema de inferência \textit{fuzzy} baseado em rede neural (ANFIS) para previsão de preço na série temporal do índice Bovespa utilizando diversas estratégias de negociação. Atuei no desenvolvimento e revisão do código, bem como fornecendo orientações relacionadas a metodologia do trabalho.\\
\textit{Habilidades utilizadas:} \textbf{MATLAB, Machine Learning, ANFIS, Shell Scripting.\\}}

\twentyitem
{Desenvolvimento de Material Didático (Em andamento):}
{'Tutorial de Simulação e Projeto de Circuitos Integrados - Uma abordagem Freeware'}
{}
{\hspace{8pt} Com o intuito de propiciar uma experiência mais prática aos alunos da matéria de Introdução ao Projeto de Circuitos Integrados, pesquisei, testei e desenvolvi um fluxo de projeto de circuitos integrados analógicos e digitais fazendo uso exclusivo de ferramentas freeware com portabilidade para múltiplos sistemas operacionais. Para facilitar o uso dessas ferramentas, estou escrevendo, em parceria com o Professor José Camargo, um tutorial que já se encontra em uso na disciplina.

\hspace{8pt}\textit{Habilidades utilizadas:} \textbf{LTspice, Spice Opus, Electric VLSI, VHDL, Circuitos Analógicos, Sistemas Digitais.\\}}

\twentyitem
{Trabalho de Conclusão de Curso}
{'Development of a Context-Aware Physical Exercise Safety Recommendation System'}
{}
{\hspace{8pt}Esse trabalho visou o desenvolvimento de um sistema capaz de adquirir informações do ambiente e do preparo físico do usuário e de, com base nessas informações, gerar uma recomendação a respeito da prática segura de atividades físicas, de forma a prevenir lesões e outras consequências negativas do exercício físico realizado em condições inadequadas. Este projeto começou a ser desenvolvido em um estágio no Laboratório de Arte Computacional (Medialab) do Instituto de Artes da UnB, orientado pela professora Suzete Venturelli, em 2012. Partes do projeto foram desenvolvidas no \textit{LPCI} em uma parceria informal através do Professor José Edil Guimarães de Medeiros e da então doutoranda Ana Régia de Mendonça Neves e no Laboratório de Robótica e Automação (LARA), orientado pelo professor Antônio Padilha Lanari Bo.

\hspace{8pt}O sistema desenvolvido foi apresentado oralmente em 2012 no 11º Encontro Internacional de Arte e Tecnologia (\#11. Art - Exposição Em-Meio \#4.0) e em 2013 no 12º Encontro Internacional de Arte e Tecnologia (\#12. Art - Exposição Em-Meio \#5.0), com publicação nos anais do congresso. Além disso, foi defendido como trabalho de conclusão de curso pelos alunos Tiago Tiveron e Matheus Bastos (IESB - 2013) e por mim (2017).

\hspace{8pt}\textit{Habilidades utilizadas: }\textbf{Java, Arduino, Fritzing, Osciloscópio, Processamento Digital de Sinais, Circuitos Analógicos, Aquisição e Condicionamento de Sinais, Processamento em Tempo Real, Filtros, Osciladores.}
}\\


\twentyitem{Projeto Final de Intercâmbio com a University of Glasgow}
{'Investigating Improvements to a simple Pulse Rate Monitor'}
{}
{\hspace{8pt}Durante os três últimos meses do meu intercâmbio na University of Glasgow, trabalhei na análise e melhoria do projeto de um oxímetro utilizado para monitoramento indireto da frequência cardíaca sob supervisão do professor Dr. David Muir. Durante este período, fiz uso de ferramentas de desenho auxiliado para computador para impressão 3d de peças mecânicas simples, projetei e fabriquei a placa de circuito impresso, e desenvolvi um software para visualização dos dados em tempo real.

\hspace{8pt}\textit{Habilidades utilizadas:} \textbf{Java, Freedom Board, Fritzing, Osciloscópio, Processamento Digital de Sinais, Circuitos Analógicos, Aquisição e Condicionamento de Sinais, Processamento em Tempo Real, Filtros.}}
\end{twenty}
\end{document} 
