%%%%%%%%%%%%%%%%%%%%%%%%%%%%%%%%%%%%%%%%%
% Twenty Seconds Resume/CV
% LaTeX Template
% Version 1.0 (14/7/16)
%
% Original author:
% Carmine Spagnuolo (cspagnuolo@unisa.it) with major modifications by 
% Vel (vel@LaTeXTemplates.com) and Harsh (harsh.gadgil@gmail.com)
%
% License:
% The MIT License (see included LICENSE file)
%
%%%%%%%%%%%%%%%%%%%%%%%%%%%%%%%%%%%%%%%%%

%----------------------------------------------------------------------------------------
%	PACKAGES AND OTHER DOCUMENT CONFIGURATIONS
%----------------------------------------------------------------------------------------

\documentclass[a4paper]{twentysecondcv} % a4paper for A4
\usepackage[utf8]{inputenc}
\usepackage[portuguese]{babel}

% Command for printing skill overview bubbles
% Programming skill bars
\programming{{Microsoft Office $-$ LibreOffice $-$ G Suite / 5},  {Linux $-$ 
\LaTeX $-$  Raspberry Pi/ 4}, {Assembly $-$ SQL $-$  Python/ 2}, {Git $-$ Linux 
Server $-$ Confluence /4},{ADALM-PLUTO $-$ Freedom Board $-$ ARM / 3}, {Java $-$ 
C \slash C++ $-$ MATLAB \slash Octave / 4}, {VHDL $-$ TCL $-$ Shell Scripting 
/ 4},{Arduino $-$ 8051 $-$ 8086 / 3}, {Mentor Graphics $-$ Synopsys $-$ Eagle 
/ 2}, {Xilinx Vivado $-$ Altera Quartus $-$ FPGA / 4}, {Electric VLSI $-$ Doxygen
$-$ TaskJuggler / 4},{Cadence $-$ LTspice $-$ SpiceOpus / 4}}

\Languages{{Italiano / 1}, {Francês / 1.3}, {Inglês / 6}, {Português (Nativo) / 
6}}

%\uCuP{{Arduino $-$ 8051 $-$ 8086 / 3}, {Mentor Graphics $-$ Synopsys $-$ Fritzing / 3}, {Xilinx Vivado $-$ Altera Quartus $-$ FPGA / 4}, {Electric VLSI $-$ Doxygen $-$ TaskJuggler / 5},{Cadence $-$ LTspice $-$ SpiceOpus / 5}}


% Projects text
\education{
\textbf{Graduate Certificate, Embedded Systems}\\
University of Colorado - Boulder\\
2020 - Em Andamento | Colorado, USA

\textbf{Beng, Electrical Engineering and Biomedical Engineering}\\
Programa de Intercâmbio - Ciência Sem Fronteiras \\
University of Glasgow \\
2014 - 2015 | Glasgow, Scotland

\textbf{BEng., Engenharia Elétrica} \\
Universidade de Brasília \\
2011 - 2019 | Brasília, Brasil
}

%----------------------------------------------------------------------------------------
%	 PERSONAL INFORMATION
%----------------------------------------------------------------------------------------
% If you don't need one or more of the below, just remove the content leaving the command, e.g. \cvnumberphone{}

\cvname{Guilherme \\Shimabuko} % Your name
\cvjobtitle{ Engenheiro Eletricista } % Job
% title/career

\cvnumberphone{+55 61 9 9841-4243} % Phone number
\cvmail{guilherme.shimabuko@colorado.edu} % Email address

%----------------------------------------------------------------------------------------

\begin{document}

\makeprofile % Print the sidebar
 
%----------------------------------------------------------------------------------------
%	 EXPERIENCE
%----------------------------------------------------------------------------------------
\section{Sobre: }
\small{Engenheiro Eletricista com forte interesse em eletrônica e computação. 
Entusiasta das filosofias de desenvolvimento de software livre e de código e 
hardware abertos, com experiência de longa data em diversas distribuições Linux. 
Pó-ativo, trabalha bem em equipe, com experiência em gestão e treinamento de 
equipes.}

\section{Experiência Relevante}
\begin{twenty} % Environment for a list with descriptions
\twentyitem
    	{LPCI - Departamento de Engenharia Elétrica - UnB}
		{Assistente de Pesquisa \hfill \textit{Janeiro de 2020 - Atualmente}\\}
        {Estagiário \hfill\textit{Novembro de 2018 - Dezembro de 2019}}
        {
            \color{pblue}{\hspace{8pt}-\hspace{4pt}} \color{black}\textbf{Sistemas
            Digitais: }Design e validação de sistemas digitais para FPGA e     
            ASIC para comunicação satelital utilizando ferramentas Cadence.
            
            \color{pblue}{\hspace{8pt}-\hspace{4pt}} \color{black}\textbf{
            Tecnologia da Informação: }Validação e manutenção das ferramentas de 
            projeto e gestão de servidores linux (Archlinus e CentOS), bem como 
            suporte para os clientes (Ubuntu, CentOS, ArchLinux e Windows).
            
            \color{pblue}{\hspace{8pt}-\hspace{4pt}} \color{black}\textbf{Recursos 
            Humanos: }Seleção e treinamento para formação de equipe de projeto de
            circuitos integrados.
            
            \color{pblue}{\hspace{8pt}-\hspace{4pt}} \color{black}\textbf{Gestão
            de Projetos: }Elaboração de planejamento de projeto. Acompanhamento de
            atividades. Implantação de sistema de log de atividades. Documentação
            de projeto. Implantação, otimização e automação de fluxo de projeto.

        }
        \\
	\twentyitem
    	{DFchip}
		{Estagiário}
        {Novembro de 2015 - Janeiro de 2017}
        {
            \color{pblue}{\hspace{8pt}-\hspace{4pt}} \color{black}{\textbf{
            Sistemas Digitais: }Design e validaçãoo de sistemas sigitais para 
            FPGA, ASIC e SoC, bem como implantação e documentação do fluxo de 
            projetos de sistemas digitais utilizando ferramentas Cadence.}
            
            \color{pblue}{\hspace{8pt}-\hspace{4pt}} \color{black}{\textbf{
            Sistemas Analógicos: }Treinamento em design de circuitos analógicos
            utilizando ferramentas Cadence.}
            
            \color{pblue}{\hspace{8pt}-\hspace{4pt}} \color{black}{\textbf{
            Tecnologia da Informação: }Validação e manutenção das ferramentas de
            projeto e gestão de clientes com sistema Linux (OpenSUSE).}
        }
        \\
    \twentyitem
    	{Medialab – Instituto de Artes – UnB}
		{Estagiário}
		{Fevereiro de 2014 - Agosto de 2014 e Julho de 2012 – Marçoo de 2013}
        {
            \color{pblue}{\hspace{8pt}-\hspace{4pt}} \color{black}\textbf{
            Prototipagem: }Desenvolvimento de protótipo de dispositivos 
            eletrônicos para exibições de Arte e Tecnologia.
        }
        \\
    \twentyitem
    {LPCI - Departamento de Engenharia Elétrica - UnB}
    {Estagiário}
    {Abril de 2013 - Janeiro de 2014}
    {
        \color{pblue}{\hspace{8pt}-\hspace{4pt}} \color{black}\textbf{
        Prototipagem: }Desenvolvimento de protótipo de dispositivos eletrônicos 
        para exibições de Arte e Tecnologia.
    }
    \\
    \twentyitem
    {ENETEC Consultoria Junior - Empresa Junior de Engenharia Elétrica}
    {Diretor de Projetos}
    {Julho de 2012 - Julho de 2013}
    {\color{pblue}{\hspace{8pt}-\hspace{4pt}} \color{black}\textbf{Gestão de 
    Projetos: }Elaboração e implantação de metodologias de gerenciamento de 
    projetos. Negociação e gerenciamento de projetos.}
    \\
    \twentyitem
    {ENETEC Consultoria Junior - Empresa Junior de Engenharia Elétrica}
    {Consultor \hfill Janeiro de 2012 - Julho de 2012\\}
    {Trainee  \hfill Julho de 2011 - Dezembro de 2011}
    {\color{pblue}{\hspace{8pt}-\hspace{4pt}} \color{black}\textbf{Consultoria: }
    Elaboração de projetos de Instalações Elétricas. Criação de Processos para a 
    Diretoria de Recursos Humanos.}
\end{twenty}

%----------------------------------------------------------------------------------------
%	 Lecturing Experience
%----------------------------------------------------------------------------------------
\section{Atividades de Docência}
\begin{twenty}

    \twentyitemshortii{Monitor de Introduçãoo ao Projeto de Circuitos Integrados}{Departamento de Engenharia Elétrica - UnB}{0/2017, 1/2018 e 2/2018}
    \twentyitemshortii{Monitor de Laboratório de Eletrônica 2}{Departamento de Engenharia Elétrica - UnB}{1/2017 e 2/2017}
    \twentyitemshortii{Instrutor de Curso de Arduíno}{Centro Acadêmico de Engenharia Elétrica - UnB}{Marçoo de 2017 - Abril de 2017}
    \twentyitemshortii{Monitor de Laboratório de Eletrônica 1}{Departamento de Engenharia Elétrica - UnB}{1/2016}
    \twentyitemshortii{Monitor de Sistemas Digitais 2}{Departamento de Engenharia
    Elétrica - UnB}{1/2014}
    \twentyitemshortii{Monitor de Laboratório de Sistemas Digitais 1}{Departamento
    de Engenharia Elétrica - UnB}{1/2013}
    \twentyitemshortii{Instrutor de Curso de Excel}{ENETEC Consultoria Junior -
    Empresa Junior de Engenharia Elétrica - UnB}{Agosto de 2012}
\end{twenty}

\newpage
\secsidebar
\section{Projetos e Publicações}
\begin{twenty}
\twentyitem
{Projeto de Pesquisa Para Parceiro Industrial:}
{'Desenvolvimento de Sistema de Comunicação Satelital baseado no padrão DVB-S2X'}
{}
{\hspace{8pt} Esse projeto visou o desenvolvimento de um sistema de comunicação
satelital móvel em parceria com uma empresa privada. A equipe da qual fiz parte 
foi responsável pela revisão e revalidação da implementação em VHDL do 
protótipo desenvolvido pela equipe de FPGA, bem como a adaptação desses códigos 
para o framework de circuitos integrados da Cadence e para os padrões de 
codificação do \textit{LPCI}. 

\hspace{8pt}Atuei como \textit{Design Leader}, treinando o grupo e gerenciando a 
equipe, revisando as entregas e projetando partes do sistema. O foco da etapa do 
projeto em que participei foi o desenvolvimento dos códigos de correção de erro, 
especialmente o algoritmo BCH. Como parte das minhas atividades, coorientei 
extra-oficialmente o trabalho de conclusão de curso do aluno Thiago Queiroz 
Holanda sob a tutela do Professor José Camargo da Costa. Esse trabalho foi 
apresentado em Dezembro de 2019 sob o título \textit{"Comparison Between the 
Implementations of a BCH DVB-S2X Decoder in FPGA and in ASIC"}. Durante esse 
período, também fui responsável pela gestão de TI do projeto. Por fim, tive uma 
breve participação na equipe de prototipagem em FPGA, trabalhando com a 
compilação cruzada de códigos C para um processador ARM.

\hspace{8pt}\textit{Habilidades utilizadas:} \textbf{VHDL, ASIC, FPGA, Cadence 
Tools, Vivado, Shell Scripting, TCL, MATLAB, Octave, Linux, Sistemas Embarcados, 
Compilação Cruzada, Buildroot, ARM, ADALM-PLUTO, SDR, Sistemas Digitais, 
Processamento Digital de Sinais.\\}}


\twentyitem
{Colaboração em Projeto de Conclusão de Curso em Machine Learning e finanças:}
{'Aplicação de inteligência computacional no mercado financeiro'}
{}
{\hspace{8pt} Como assistente de pesquisa do LPCI, colaborei com o desenvolvimento
do trabalho de conclusão de curso do aluno Tainan Soares Rodrigues, orientado pelo
professor Alexandre Romariz e apresentado em Dezembro de 2019. Esse projeto visou 
a avaliação da eficácia de um sistema de inferência \textit{fuzzy} baseado em rede
neural (ANFIS) para previsão de preço na série temporal do Índice Bovespa 
utilizando diversas estratégias de negociação. Atuei no desenvolvimento e revisão
do código, bem como fornecendo orientações relacionadas a metodologia do trabalho.
\\
\textit{Habilidades utilizadas:} \textbf{MATLAB, Machine Learning, ANFIS, 
Shell Scripting.\\}}

\twentyitem
{Desenvolvimento de Material Didático (Em andamento):}
{'Tutorial de Simulação e Projeto de Circuitos Integrados - Uma abordagem 
Freeware'}
{}
{\hspace{8pt} Com o intuito de propiciar uma experiência mais prática aos alunos 
da matéria de Introdução ao Projeto de Circuitos Integrados, pesquisei, testei e 
desenvolvi um fluxo de projeto de circuitos integrados analógicos e digitais 
fazendo uso exclusivo de ferramentas freeware com portabilidade para múltiplos 
sistemas operacionais. Para facilitar o uso dessas ferramentas, escrevi, 
em parceria com o Professor José Camargo, um tutorial que já se encontra em uso 
na disciplina.

\hspace{8pt}\textit{Habilidades utilizadas:} \textbf{LTspice, Spice Opus, 
Electric VLSI, VHDL, Circuitos Analógicos, Sistemas Digitais.\\}}

\twentyitem
{Trabalho de Conclusão de Curso}
{'Development of a Context-Aware Physical Exercise Safety Recommendation System'}
{}
{\hspace{8pt}Esse trabalho visou o desenvolvimento de um sistema capaz de adquirir 
informações do ambiente e do preparo físico do usuário e de, com base nessas 
informações, gerar uma recomendação a respeito da prática segura de atividades 
físicas, de forma a prevenir lesões e outras consequências negativas do exercício 
físico realizado em condições inadequadas. Este projeto começou a ser desenvolvido
em um estágio no Laboratório de Arte Computacional (Medialab) do Instituto de 
Artes da UnB, orientado pela professora Suzete Venturelli, em 2012. Partes do 
projeto foram desenvolvidas no \textit{LPCI} em uma parceria informal através do 
Professor José Edil Guimarães de Medeiros e da então doutoranda Ana Régia de 
Mendonça Neves e no Laboratório de Robótica e Automação (LARA), orientado pelo 
professor Antônio Padilha Lanari Bo.

\hspace{8pt}O sistema desenvolvido foi apresentado oralmente em 2012 no 
$11\degree$ Encontro Internacional de Arte e Tecnologia (\#11. Art - Exposição 
Em-Meio \#4.0) e em 2013 no $12\degree$ Encontro Internacional de Arte e 
Tecnologia (\#12. Art - Exposição Em-Meio \#5.0), com publicação nos anais do 
congresso. Além disso, foi defendido como trabalho de conclusão de curso pelos 
alunos Tiago Tiveron e Matheus Bastos (IESB - 2013) e por mim (2017).

\hspace{8pt}\textit{Habilidades utilizadas: }\textbf{Java, Arduino, Fritzing,
Osciloscópio, Processamento Digital de Sinais, Circuitos Analógicos, Aquisição e 
Condicionamento de Sinais, Processamento em Tempo Real, Filtros, Osciladores.}
}\\


\twentyitem{Projeto Final de Intercâmbio com a University of Glasgow}
{'Investigating Improvements to a simple Pulse Rate Monitor'}
{}
{\hspace{8pt}Durante os três últimos meses do meu intercâmbio na University of 
Glasgow, trabalhei na análise e melhoria do projeto de um oxímetro utilizado para
monitoramento indireto da frequência cardíaca sob supervisão do professor Dr. 
David Muir. Durante este período, fiz uso de ferramentas de desenho auxiliado para
computador para impressão 3d de peças mecânicas simples, projetei e fabriquei a 
placa de circuito impresso, e desenvolvi um software para visualização dos dados 
em tempo real.

\hspace{8pt}\textit{Habilidades utilizadas:} \textbf{Java, Freedom Board, 
Fritzing, Osciloscópio, Processamento Digital de Sinais, Circuitos Analógicos, 
Aquisição e Condicionamento de Sinais, Processamento em Tempo Real, Filtros.}}
\end{twenty}
\end{document} 
