%%%%%%%%%%%%%%%%%%%%%%%%%%%%%%%%%%%%%%%%
% Twenty Seconds Resume/CV
% LaTeX Template
% Version 1.0 (14/7/16)
%
% Original author:
% Carmine Spagnuolo (cspagnuolo@unisa.it) with major modifications by 
% Vel (vel@LaTeXTemplates.com) and Harsh (harsh.gadgil@gmail.com)
%
% License:
% The MIT License (see included LICENSE file)
%
%%%%%%%%%%%%%%%%%%%%%%%%%%%%%%%%%%%%%%%%%

%----------------------------------------------------------------------------------------
%	PACKAGES AND OTHER DOCUMENT CONFIGURATIONS
%----------------------------------------------------------------------------------------

\documentclass[a4paper]{twentysecondcv} % a4paper for A4
\usepackage[utf8]{inputenc}
\usepackage[portuguese]{babel}

% Command for printing skill overview bubbles
% Programming skill bars
\programming{{Microsoft Office $-$ LibreOffice $-$ G Suite / 5},  {Linux $-$ \LaTeX $-$  Raspberry Pi/ 4}, {Assembly $-$ SQL $-$  Python/ 2}, {Git $-$ Linux Server $-$ Confluence /4},{ADALM-PLUTO $-$ Freedom Board $-$ ARM / 3}, {Java $-$ C \slash C++ $-$ MATLAB \slash Octave / 4}, {VHDL $-$ TCL $-$ Shell Scripting / 4}, {Arduino $-$ 8051 $-$ 8086 / 3}, {Mentor Graphics $-$ Synopsys $-$ Eagle / 2}, {Xilinx Vivado $-$ Altera Quartus $-$ FPGA / 4}, {Electric VLSI $-$ Doxygen $-$ TaskJuggler / 4},{Cadence $-$ LTspice $-$ SpiceOpus / 4}}

\Languages{{Italian / 1}, {French / 1.3}, {English / 6}, {Portuguese (Native) / 6}}

%\uCuP{{Arduino $-$ 8051 $-$ 8086 / 3}, {Mentor Graphics $-$ Synopsys $-$ Fritzing / 3}, {Xilinx Vivado $-$ Altera Quartus $-$ FPGA / 4}, {Electric VLSI $-$ Doxygen $-$ TaskJuggler / 5},{Cadence $-$ LTspice $-$ SpiceOpus / 5}}


% Projects text
\education{
\textbf{Graduate Certificate, Embedded Systems}\\
University of Colorado - Boulder\\
2020 - Ongoing | Colorado, USA

\textbf{Beng, Electrical Engineering and Biomedical Engineering}\\
Exchange Programme - Science Without Borders \\
University of Glasgow \\
2014 - 2015 | Glasgow, Scotland

\textbf{BEng., Electrical Engineering} \\
University of Brasilia \\
2011 - 2019 | Bras�lia, Brazil
}

%----------------------------------------------------------------------------------------
%	 PERSONAL INFORMATION
%----------------------------------------------------------------------------------------
% If you don't need one or more of the below, just remove the content leaving the command, e.g. \cvnumberphone{}

\cvname{Guilherme \\Shimabuko} % Your name
\cvjobtitle{ Electrical Engineer } % Job
% title/career

\cvnumberphone{+55 61 9 9841-4243} % Phone number
\cvmail{gshimabuko@ieee.org} % Email address

%----------------------------------------------------------------------------------------

\begin{document}

\makeprofile % Print the sidebar
 
%----------------------------------------------------------------------------------------
%	 EXPERIENCE
%----------------------------------------------------------------------------------------
\section{About: }
\small{Electrical Engineer with strong interest in Electronics and Computer Systems. Open Source and Open Hardware Enthusiast with long term experience in several Linux distributions. Proactive, team-player, and with experience in management and team training.}

\section{Relevant Experience}
\begin{twenty} % Environment for a list with descriptions
\twentyitem
    	{LPCI - Department of Electrical Engineering - UnB}
		{Research Assistant \hfill \textit{January, 2020 - Currently}\\}
        {Intern \hfill\textit{November, 2018 - December, 2019}}
        {
            \color{pblue}{\hspace{8pt}-\hspace{4pt}} \color{black}\textbf{Digital Systems: }Digital circuits design and validation on FPGA and ASIC for satellite communications using Cadence IC Design Framework.
            
            \color{pblue}{\hspace{8pt}-\hspace{4pt}} \color{black}\textbf{IT Management: }Validation and maintenance of project tools. Linux Server management, as well as client support (Ubuntu, CentOS, ArchLinux).
            
            \color{pblue}{\hspace{8pt}-\hspace{4pt}} \color{black}\textbf{Human Resources: }Team recruiting and training in order to assemble an integrated circuits design team.
            
            \color{pblue}{\hspace{8pt}-\hspace{4pt}} \color{black}\textbf{Project Management: }Project planning and tracking. Activity log system implementation. Project Documentation. Implementation, optimization and automation of design flow.
        }
        \\
	\twentyitem
    	{DFchip}
		{Intern}
        {November, 2015 - January, 2017}
        {
            \color{pblue}{\hspace{8pt}-\hspace{4pt}} \color{black}{\textbf{Digital Systems: }Digital circuits design and validation on FPGA, ASIC, and SoC, as well as the implementation and documentation of Digital Systems design flow using Cadence IC Design Framework.}
            
            \color{pblue}{\hspace{8pt}-\hspace{4pt}} \color{black}{\textbf{Analogue Systems: }Training in Analog Circuit Design using Cadence IC Design Framework.}
            
            \color{pblue}{\hspace{8pt}-\hspace{4pt}} \color{black}{\textbf{IT Management: }Validation and maintenance of Design Flow Tools in Linux Clients (OpenSUSE).}
        }
        \\
    \twentyitem
    	{Medialab - Institute of Art - UnB}
		{Intern \hfill February, 2014 - August 2014\\}
		{Intern \hfill July, 2012 - March, 2013}
        {
            \color{pblue}{\hspace{8pt}-\hspace{4pt}} \color{black}\textbf{Prototyping: }Development of software and hardware for ubiquitous technology system prototypes for Arts and Technology Conferences and Exhibits.
        }
        \\
    \twentyitem
    {LPCI - Department of Electrical Engineering - UnB}
    {Intern}
    {April, 2013 - January, 2014}
    {
        \color{pblue}{\hspace{8pt}-\hspace{4pt}} \color{black}\textbf{Prototyping: }Development of software and hardware for ubiquitous technology system prototypes for Arts and Technology Conferences and Exhibits.
    }
    \\
    \twentyitem
    {ENETEC Junior Consultancy - Electrical Engineering Junior Enterprise}
    {Projects Director}
    {July, 2012 - July, 2013}
    {\color{pblue}{\hspace{8pt}-\hspace{4pt}} \color{black}\textbf{Project Management: }Implementation of Projects Management methodology and processes. Project negotiation, planning and tracking.}
    \\
    \twentyitem
    {ENETEC Junior Consultancy - Electrical Engineering Junior Enterprise}
    {Consultant \hfill January, 2012 - July, 2012\\}
    {Trainee  \hfill July, 2011 - December, 2011}
    {\color{pblue}{\hspace{8pt}-\hspace{4pt}} \color{black}\textbf{Consulting: }Electrical projects elaboration. Elaboration, execution and maintenance of Human Resources Processes.}
\end{twenty}

%----------------------------------------------------------------------------------------
%	 Lecturing Experience
%----------------------------------------------------------------------------------------
\section{Tutoring and Lecturing}
\begin{twenty}

    \twentyitemshortii{Introduction to Integrated Circuits Projects Teacher Assistant}{Electrical Engineering Department - UnB}{January, 2017 - March, 2017 and March, 2018 - December, 2018}
    \twentyitemshortii{Electronics 2 Laboratory Teacher Assistant}{Electrical Engineering Department - UnB}{March, 2017 - December, 2017}
    \twentyitemshortii{Arduino Course Instructor}{Electrical Engineering Students Union - UnB}{March, 2017 - April de 2017}
    \twentyitemshortii{Electronics 1 Laboratory Teacher Assistant}{Department of Electrical Engineering - UnB}{March, 2016 - July, 2016}
    \twentyitemshortii{Digital Systems 2 Teacher Assistant}{Department of Electrical Engineering - UnB}{March, 2014 - July, 2014}
    \twentyitemshortii{Digital Systems 1 Laboratory Teacher Assistant}{Department of Electrical Engineering - UnB}{March, 2013 - July, 2013}
    \twentyitemshortii{Excel Course Instructor}{ENETEC Junior Consultancy - Electrical Engineering Junior Enterprise - UnB}{August, 2012}
\end{twenty}

\newpage
\secsidebar
\section{Projects and Publications}
\begin{twenty}
\twentyitem
{Research Project with Industrial Partner}
{'DVB-S2X Standard Based Satellite Communications System Development'}
{}
{\hspace{8pt} This project aimed at developing a satellite communication system based on the DVB-S2X standard, and was developed in partnership with an industrial party. My team was responsible for reviewing and re-validating the VHDL implementation delivered by an FPGA team, and then adapting these source codes to the laboratory guidelines. We were also responsible for presenting a comprehensive analysis and resource estimation of an ASIC implementation using the Cadence IC Design Framework. 

\hspace{8pt}I acted as the \textit{Design Leader}, training and managing the team, reviewing the deliverables and also designing some of the system. The focus of the phase I took part was the implementation of error correction blocks, specially the BCH algorithm. As part of my activities, I co-advised the student Thiago Queiroz Holanda in his final graduation project under tutelage of Professor José Camargo da Costa. This work was presented to an assessment committee in December, 2019 with the title \textit{"Comparison Between the Implementations of a BCH DVB-S2X Decoder in FPGA and in ASIC"}. During this period, I was also responsible for the IT management. I also had a brief participation in the FPGA team working with the cross-compilation of embedded code for an ARM processor.

\hspace{8pt}\textit{Employed Abilities: } \textbf{VHDL, ASIC, FPGA, Cadence Tools, Vivado, Shell Scripting, TCL, MATLAB, Octave, Linux, Embedded Systems, cross compilation, Buildroot, ARM, ADALM-PLUTO, SDR, Digital Systems, Digital Signals Processing.\\}}


\twentyitem
{Final Project Collaboration in Machine Learning Applied to the Financial Market:}
{'Computational Intelligence Application in the Financial Market:'}
{}
{\hspace{8pt} As a research assistant at LPCI, I collaborated with the student Tainan Soares Rodrigues in the development of his final project under the tutelage of Professor Alexandre Romariz. This project was presented to an assessment committee in December, 2019.

This Project aimed at assessing the efficacy of an Adaptive Neuro-Fuzzy Inference System in estimating future prices in the temporal series of the Bovespa Index using several trading strategies. I took part in the development and revision of the source code as well as providing guidance related to the work methodology.\\
\textit{Employed Abilities:} \textbf{MATLAB, Machine Learning, ANFIS, Shell Scripting.\\}}

\twentyitem
{Educational Material Production (Ongoing):}
{'Integrated Circuits Project and Simulation Tutorial - A Freeware Approach'}
{}
{\hspace{8pt} Aiming at a more practical approach for the Introduction to Integrated Circuits Project course, I researched, tested and developed a design flow making use of free software only, with portability to multiple operational systems. In order to facilitate the use of these tools, a tutorial was developed and is already being used in the laboratory sessions of this course lectured by Professor José Camargo da Costa.

\hspace{8pt}\textit{Employed Abilities:} \textbf{LTspice, Spice Opus, Electric VLSI, VHDL, Analog Circuits, Digital Systems.\\}}

\twentyitem
{Research Project and Final Project}
{'Development of a Context-Aware Physical Exercise Safety Recommendation System'}
{}
{\hspace{8pt}This project was aimed at the development of a system capable of acquiring environmental information and the user's fitness conditions and then, with that information, generate a recommendation regarding the safe practice of physical activities in order to prevent injury and other possible negative consequences of inadequate conditions to exercise. The project was developed in an internship at the Computational Art Laboratory (\textit{Medialab}), at UnB, under tutelage of Professor Suzete Venturelli, in 2012. Other parts of the project were developed in the Integrated Circuits Project Laboratory (\textit{LPCI}) in an informal partnership with Professor José Edil Guimarães de Medeiros and the former PhD Candidate, now Professor, Ana Régia de Mendonça Neves. Later, I was also under advisement of Professor Antônio Padilha Lanari Bo at the Automation and Robotics Laboratory (\textit{LARA}).

\hspace{8pt}An oral presentation of the system was made in 2012 in the $11^{th}$ International Meeting of Art and Technology (\#11. Art - Exposi��o Em-Meio \#4.0) and in 2013 in the $12^{nd}$ International Meeting of Art and Technology (\#12. Art - Exposição Em-Meio \#5.0), when an article was also published in the Congress Proceedings. It was also presented as my final project, in 2017 at UnB, and by the students Tiago Tiveron and Matheus Bastos in 2013 at IESB.

\hspace{8pt}\textit{Employed Abilities: }\textbf{Java, Arduino, Fritzing, Osciloscope, Digital Signals Processing, Analog Circuits, Signals Acquisition and Conditioning, Real Time Processing, Filters, Oscillators.}
}\\


\twentyitem{Exchange Programme Final Project - University of Glasgow}
{'Investigating Improvements to a simple Pulse Rate Monitor'}
{}
{\hspace{8pt}During the last three months of my exchange programme at the University of Glasgow, I worked in the analysis and improvement of a pulse rate monitor under the supervision of Dr. David Muir. In this project, I worked with 3D printed prototypes to stabilise the sensor, designed a circuit and its PCB, and I also developed embedded software to process the signal, as well as a real time data visualisation software.

\hspace{8pt}\textit{Employed Habilities:} \textbf{Java, Freedom Board, Fritzing, Osciloscope, Digital Signal Processing, Analog Circuits, Signal Acquisiton and Conditioning, Real Time Processing, Embedded Software, Filters.}}
\end{twenty}
\end{document} 
